\documentclass[12pt]{article}

\title{On the Connection between Reading and Writing}
\author{Robert Krency}
\date{\today}

\usepackage{amsmath}
\usepackage{graphicx}
\usepackage{tabularx}

% Geometry 
\usepackage{geometry}
\geometry{letterpaper, left=1in, top=1in, right=1in, bottom=1in}

% Add vertical spacing to tables
\renewcommand{\arraystretch}{1.4}

% Use Double Spacing for the paper
\usepackage{setspace}
\doublespacing

% Add space between columns
\setlength\columnsep{30pt}

% Begin Document
\begin{document}

\maketitle

The written transmission of information is one of the most influential ideas of humankind, akin to ideas such as fire, farming, or the wheel.
Writing and reading are implicitly intertwined, but this connection manifests itself in the participants; the writer to share, the reader to consume.
The writer and the reader each has responsibilities in relation to the written work.
These responsibilities give rise to the discussion of ethics then in the transfer of knowledge.

The direct result of a written work is the sharing of information.
The writer has a collection of thoughts, and wishes to dispense of it to an audience.
By participating in a matured, formal structure the writer is able to perform this ritual of knowledge transfer to the audience.
The audience is then able to utilize the structure of the work to understand the knowledge in a meaningful way.
This does result in each written work having the notion of effectiveness: does it carry out its purpose in a meaningful way?
A written work that has no structure and is unable to commit its ideas into a cohesive piece often has little value.
The effective transfer of information is vital between persons, and often without tampering or misconstruing.

This idea then gives rise to certain responsibilities that each of the writer and the reader then has.
The writer should commit to following rules and structure.
The reader then has the responsibility to read the work within its given rules, structure, and context.
The written work establishes a link between the writer and the audience, allowing the audience a glimpse into the thoughts of the writer.

However, the notion that this should be done faithfully and without misconstruing ideas is a result of societal ethics.
In this current age of disinformation, writers and readers alike are often unfaithful to the written work.
Information may be used out of context, or presented in such a way to contradict its original purpose.
The writer's purpose is not always to be truthful or consistent, and may sometimes be to deceive.
This gives the reader the responsibility of skepticism, to recognize the possibility the intent of deception.
This results then in the reader having the responsibility of bringing tools of analysis to any transfer of knowledge.

The relationship between the reader and the writer then follows an unwritten contract.
Each has responsibilities when approaching a written work.
The writer when sharing information agrees to adhere to a structure and to effectively share information.
The reader then must not only consume the information, but must do so within the given structure and context.
It is this unwritten contract that binds together the writer and reader, and ultimately reading and writing.

\end{document}