\documentclass[12pt]{article}

\title{On the Connection between Reading and Writing}
\author{Robert Krency}
\date{\today}

\usepackage[utf8]{inputenc}
\usepackage{amsmath}
\usepackage{graphicx}
\usepackage{tabularx}

% Geometry 
\usepackage{geometry}
\geometry{letterpaper, left=1in, top=1in, right=1in, bottom=1in}

% Add vertical spacing to tables
\renewcommand{\arraystretch}{1.4}

% Use Double Spacing for the paper
\usepackage{setspace}
\doublespacing

% Add space between columns
\setlength\columnsep{30pt}

% Begin Document
\begin{document}

\section{The Writing Process}

The writing process consists of the following steps, that tend to overlap and be repeated in varying orders:

\begin{tabular}{c | c}
    Active Reading & Active Writing \\ \hline
    Pre-Reading & Planning \\
    Close Reading & Drafting \\
    Post Reading & Revising \\
\end{tabular}

\subsection{Pre-Reading}

Prereading leads us to ask question: Who? What? When? Where? Why? How?

Questions lead us to answers which lead to more questions, and for the foundation of a written work.

\subsubsection{Example: Camping Out by Hemingway}

\begin{itemize}
    \item Who? Primarily men
    \item What? Camping
    \item Where? Newspaper in Toronto
    \item When? June, 1920
    \item Why?
    \item How?
\end{itemize}

When first skimming a book, what sections are important to recognize?
The \textbf{index} is a reference to find sections in the text where a topic is discussed.


\subsection{Close Reading \& Post Reading}

Do our perceptions of the work change as we read?
What do we understand or change during later readings?

When marking up a paper, leave notes on why certain bits are marked.
Why did I underline this?

Write a summary when finished reading.

Consider the three facets:
\begin{enumerate}
    \item Audience
    \item Purpose
    \begin{itemize}
        \item to Persuade
        \item to Inform
        \item to Entertain
    \end{itemize}
    \item Tone
\end{enumerate}

\section{Active Writing}

\subsection{Planning}

\begin{itemize}
    \item Brainstorm
    \item Freewrite
    \item Outline
    \item Map
    \item Cluster
\end{itemize}

\pagebreak

\subsection{Example: Choose your favorite word}

\subsubsection{Brainstorming}
\begin{itemize}
    \item Does this mean the word/phrase itself?
    \begin{itemize}
        \item Bookkeeper, Strength
    \end{itemize}

    \item Does it represent its concept?
    \begin{itemize}
        \item Palindromes
    \end{itemize}

    \item Catchphrase?
    \begin{itemize}
        \item Everybody panic!
    \end{itemize}
\end{itemize}

\subsubsection{Freewriting}

The concept of a favorite word is an odd idea.
The concept itself presupposes that a word is more than utilitarian, that it might be sentimental or have meaning to a specific person.
This brings to mind then three different questions relating to the idea of a favorite word.

The first of these questions is: is the word itself, free of context and only equipped with its dictionary meaning, a favorite?
Certain qualities that make a word stand out amongst its peers would be perhaps the spelling or an idea related to its structure.
Perhaps the word has certain meaning or sentimental value, as though it was a shared secret passphrase between siblings.
For instance, ``bookkeeper'' has the odd distinction of 3 repeated pairs of characters in a row.
Its variants are the only such words that demonstrate this feature.

\begin{itemize}
    \item Notable bits to reuse:
    \begin{itemize}
        \item concept itself presupposes there exist features that make a word standout
        \item the word itself, free of context and only equipped with its dictionary meaning
        \item ``bookkeeper'' and its variants
        \item shared secret passphrase
    \end{itemize}

\end{itemize}

\subsubsection{Write an informative paragraph}

Is it possible to favor one word over any other?
Words are equipped only with a dictionary meaning, lacking any context.
The concept of a favorite word then presupposes that there exists features that make a word worthy above its peers.
For instance, the structure of a word can bend the attention of a curious sort such as myself.
The word ``bookkeep'', and its variants, is notable for its three pairs of repeated characters uninterrupted.
To my knowledge, this is the only such instance of this pattern in the English language.


\subsubsection{Write a persuasive paragraph}

Having a favorite word may seem an odd idea at first glance.
However, its usage can be of great importance, perhaps as a great conversation piece or shared secret passphrase.
Take the word ``bookkeep'' for instance. 
The structure of this word, and its variants, is unique amongst the modern English language.
It features three pairs of characters uninterrupted, filling this writer's ideas of the previously shared scenarios.
The structure can lead to an interesting conversation about finding similar words with repeated characters.
Or perhaps its blandness makes it interesting for a secret passphrase, as it does not stand out amongst funny sounding words such as boondoggle that cannot be easily placed in everyday conversation.


\subsubsection{Write an entertaining paragraph}

Travel back to your youth, a time of freedom and fantasy.
For many, a secret clubhouse or treehouse often marked a rite of passage amongst peers or older siblings.
And of course, every secret clubhouse needs a secret passphrase.
This favored word then would need to be instantly recognizable to those in the know, and unassuming to those not.
The word ``bookkeep'' for instance is easily slotted into bland conversation that makes most slip away to their mental happy place.
However, its structure of three repeated characters uninterrupted makes it an interesting candidate.
This writer imagines a challenge to join a treehouse society, in which a riddle can be concocted.
``Three pairs are we, two vowels of secrecy, intruded by \text{?`}Que? ''


\subsubsection{Read all three paragraphs. Describe which one you like best and why.}

I have to pat myself on the back for that little riddle; though it is rather difficult.
The informative paragraph is the most direct in discussing the paragraph.
The idea of a favorite word is a personal choice, as are favorites by nature, thus making a persuasive argument somewhat futile.
An entertaining paragraph is more difficult, but a story about using the word is interesting.
Overall, I think the informative paragraph suits the word better as the concept of the favorite word is defined based on its structure, 
the personal meaning being indirect as it requires the predisposition towards patterns.


\subsubsection{Why did you choose the word that you did? What value did you find the assignment?}

The structure of the word makes it unique with its pattern of three pairs of characters uninterrupted.
This assignment is a good demonstration in the preconceived notions the writer has about what makes up a different audience and how to write to each.
Writing informatively, persuasively, or entertainingly require different structure and tone.
A single topic can be written about it each of the styles with varying outcomes, perhaps some more effective than others.



\end{document}