\documentclass[12pt]{article}

\title{On the Connection between Reading and Writing}
\author{Robert Krency}
\date{\today}

\usepackage{amsmath}
\usepackage{graphicx}
\usepackage{tabularx}

% Geometry 
\usepackage{geometry}
\geometry{letterpaper, left=1in, top=1in, right=1in, bottom=1in}

% Add vertical spacing to tables
\renewcommand{\arraystretch}{1.4}

% Use Double Spacing for the paper
\usepackage{setspace}
\doublespacing

% Add space between columns
\setlength\columnsep{30pt}

% Begin Document
\begin{document}

\section{The Writing Process}

The writing process consists of the following steps, that tend to overlap and be repeated in varying orders:
\begin{tabular}{c | c}
    Active Reading & Active Writing \\ \hline
    Pre-Reading & Planning \\
    Close Reading & Drafting \\
    Post Reading & Revising \\
\end{tabular}

\subsection{Pre-Reading}

Prereading leads us to ask question: Who? What? When? Where? Why? How?

Questions lead us to answers which lead to more questions, and for the foundation of a written work.

\subsubsection{Example: Camping Out by Hemingway}

\begin{itemize}
    \item Who? Primarily men
    \item What? Camping
    \item Where? Newspaper in Toronto
    \item When? June, 1920
    \item Why?
    \item How?
\end{itemize}

When first skimming a book, what sections are important to recognize?
The \textbf{index} is a reference to find sections in the text where a topic is discussed.


\subsection{Close Reading & Post Reading}

Do our perceptions of the work change as we read?
What do we understand or change during later readings?

When marking up a paper, leave notes on why certain bits are marked.
Why did I underline this?

Write a summary when finished reading.

Consider the three facets:
\begin{enumerate}
    \item Audience
    \item Purpose
    \begin{itemize}
        \item to Persuade
        \item to Inform
        \item to Entertain
    \end{itemize}
    \item Tone
\end{enumerate}



\end{document}