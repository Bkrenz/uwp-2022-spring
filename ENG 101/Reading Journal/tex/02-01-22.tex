\documentclass[journal.tex]{subfiles}

\usepackage{tikz}
\usepackage{amsmath}
\usepackage{graphicx}
\usepackage{tabularx}
\usepackage{multicol}
\usepackage{algpseudocode}
\usepackage{algorithm}

% Add vertical spacing to tables
\renewcommand{\arraystretch}{1.4}

% Begin Document
\begin{document}

\pagebreak
\section*{February 1, 2022}

We examined a piece written by Ben Healy today about gossip.
This piece was interesting because it takes something that often has a negative connotation and flips that assertion.
Essentially Ben Healy is arguing that gossip is good, and provides evidence.

It's a good example of starting with looking at the information about the author and text.
Looking into who wrote it and where it was published reveals its audience, which is targeted at more educated people.
That knowledge can mean different things to different people (ie, liberal or snooty or upper class or whatever).

This does form the goalposts for Healy's argument and the way it must be constructed.
It uses an evidence based format, backing up a these of three separate reasons why gossip would be good.
These are generally done with statistical studies, by experts in their fields.
Healy namedrops some very well known and prestigious universities, which his audience would regard the studies as very reliable.

This type of article is appealing to me, as evidence and numbers are generally unbiased.
It's also a good reminder to look into the claims and make sure they are not taken out of context.
However, the reputation and credentials of Healy should allow the reader to believe in the argument.

Perhaps I will gossip more.

\end{document}