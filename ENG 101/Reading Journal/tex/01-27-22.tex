\documentclass[journal.tex]{subfiles}

\usepackage{tikz}
\usepackage{amsmath}
\usepackage{graphicx}
\usepackage{tabularx}
\usepackage{multicol}
\usepackage{algpseudocode}
\usepackage{algorithm}

% Add vertical spacing to tables
\renewcommand{\arraystretch}{1.4}

% Begin Document
\begin{document}

\pagebreak
\section*{January 27, 2022}

Missed class.

The readings were focused on the initial stages of writing a paper, Discovery and Drafting.
It can be difficult to take an idea or prompt and turn it into a solid, focused work.

For me, this will mainly come through asking questions.
My personal preference is writing a paper focused on a specific topic, something more informative.
Having an open ended, creative, vague prompt leaves me usually lost.

It's important to continue questioning your notes to find more specific answers and more detailed information about the prompt.
Every relevant point should be justified and consistent.
It's important to keep within scope, however, which is something that can easily be missed and led astray.
This is the time to lay the foundations of a paper, and gain knowledge of the subject area.

It occurs to me that my idea of papers is mostly focused around a more informative, research style.
Some of these ideas of preparation that I have cannot be applied in the same way to a more creative or fiction work.

\end{document}