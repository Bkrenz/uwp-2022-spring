\documentclass[journal.tex]{subfiles}

\usepackage{tikz}
\usepackage{amsmath}
\usepackage{graphicx}
\usepackage{tabularx}
\usepackage{multicol}
\usepackage{algpseudocode}
\usepackage{algorithm}

% Add vertical spacing to tables
\renewcommand{\arraystretch}{1.4}

% Begin Document
\begin{document}

\pagebreak
\section*{January 25, 2022}

The focus of class today was on turning an idea into a written work.
This was through the common idea of a ``favorite word'' and each chose one.
I began this assignment exploring what it meant to be a favorite word.
Eventually, I decided on `bookkeep' and the reasons for it.

We explored free writing, a quick revisions pass, and then forming the idea into three separate styles: persuasive, informational, and entertaining.
The informational is perhaps the easiest to write and my preferred style.
Persuasive was a bit more difficult as I found the entire concept to be rooted in a deep, personal preference akin to belief, and am reminded of a quote:
<something something about convincing someone a thing built on reason when his understanding has no reason>
The entertaining paragraph was mostly fun and I concocted a story around it.

It was interesting to consider how different people would interpret the concept of a favorite word.
I focused on the word itself, free of its context and definition.
The structure of the word and its unique property is what drove me to `bookkeep'.

The purpose of writing the different paragraphs was to explore different tones and audience that a writer would engage.
The progression of going through each of these styles and techniques helps my writing.

\end{document}